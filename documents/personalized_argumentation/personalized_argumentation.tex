\documentclass[11pt]{article}
\usepackage{acl2016}
\usepackage{times}
\usepackage{url}
\usepackage{latexsym}

\usepackage{graphicx}

\usepackage[fleqn]{amsmath}
\usepackage{amssymb}
\usepackage{amstext}
\usepackage{amsfonts}
\usepackage{amsthm}

\usepackage{cite}
\usepackage{algorithm2e}
\usepackage{array}
\usepackage[caption=false,font=footnotesize]{subfig}
\usepackage{url}
\usepackage{tabularx}

\newcommand{\bs}{\boldsymbol}  
\newcommand{\wrtd}{\mathrm{d}}

\makeatletter
\makeatother %some sort of hack related to the symbol @

\DeclareMathOperator*{\argmax}{\arg\!\max\!} %argmax operator

%%%%%%%%%%%%%%%%%%%%%%%%%%%%%%%%%%%%%%%%%%%%%%%%%%%%%%%%%%%%%%%%%

\title{ 
Personalised Models of Argument Convincingness
}

\author{\IEEEauthorblockN{Anonymous}
\IEEEauthorblockA{Anonymous, \\
Anonymous \\
Email: anonymous}
\and
\IEEEauthorblockN{Anonymous}
\IEEEauthorblockA{Anonymous, \\
Anonymous \\
Email: anonymous}
\and
\IEEEauthorblockN{Anonymous}
\IEEEauthorblockA{Anonymous, \\
Anonymous \\
Email: anonymous}
}

\begin{document}

\maketitle

\begin{abstract}
\end{abstract}

% For peer review papers, you can put extra information on the cover
% page as needed:
% \ifCLASSOPTIONpeerreview
% \begin{center} \bfseries EDICS Category: 3-BBND \end{center}
% \fi
%
% For peerreview papers, this IEEEtran command inserts a page break and
% creates the second title. It will be ignored for other modes.
%\IEEEpeerreviewmaketitle

%%%%%%%%%%%%%%%%%%%%%%%%%%%%%%%%%%%%%%%%%%%%%%%%%%%%%%%%%%%%%%%%%

\section{Introduction}\label{sec:intro}

We hypothesise that different people find different types of argument more convincing 
than others and therefore, 
textual features have varying levels of importance in determining convincingness, 
depending on the audience. 
We investigate whether certain combinations of textual features are indicative of an argument's convincingness to a particular person.
We hypothesise that predictions of convincingness will be more accurate if we adapt the model to the individual reader based on their previously observed preferences. 
However, preference data for a single individual for any given task can be very sparse, 
so it will be necessary to consider the similarities between different users' preferences.
Furthermore, the  computational cost of learning independent models for each person and each task may be
impractically high, suggesting a need for more efficient approaches that combine information from multiple users.

Our approach is therefore to identify correlations between different people's preferences
so that we can learn shared models of convincingness that can then be adapted to individuals to improve predictions of argument convincingness. 
We aim to establish whether such a model can be learned by observing pairwise convincingness preferences, 
%and whether language features extracted from arguments can further improve performance when which argument a person will find most convincing. 

The experiments evaluate a number of techniques for modelling worker preferences, different types of language features, and the correlations between workers and features. 
We investigate whether workers with similar preferences according to each model give similar justifications for their decisions, thereby lending additional support for models based on correlations between preferences.

We provide a new preference learning model to handle large numbers of potentially very sparse features and large numbers of people. Our Bayesian approach enables us to 
perform automatic feature selection, learn in semi-supervised or unsupervised modes, 
and fully account for model and parameter uncertainty, while scaling to large numbers of input features. 

\section{Related Work}\label{sec:related}

The Gaussian process (GP) preference learning approach of \cite{chu2005preference} resolves such inconsistencies and provides a way to predict rankings or preferences for 
items for which we have not observed any pairwise comparisons based on the item's features. 
An extension to multiple users was proposed by \cite{houlsby2012collaborative}, 
but this method suffered from poor scalability.

Matrix factorisation techniques are commonly used in recommender systems to discover latent
user and item features but can fail if the
data is very sparse unless suitably regularised or given a Bayesian treatment.
Matrix factorisation techniques are also unsuitable for pairwise comparisons as they 
must be learned using explicit numerical ratings.
A more scalable approach that incorporates probabilistic matrix factorisation
(specifically, probabilistic PCA) was proposed by \cite{khan2014scalable}.
Their method is applicable to both pairwise comparisons and ratings data
and as such could be used to learn the model from implicit feedback such as clicks on an item. However, it may be more suitable to use a model for such feedback that explicitly considers the different bias and noise of each type or source of feedback. For such
a purpose, the model of \cite{dawid_maximum_1979} may be appropriate but has to date
been used for classifier combination and categorical labelling tasks in crowdsourcing and has not been applied to preference learning from different types of feedback. 
Bayesian approaches are suited to handling these problems of data sparsity, noise and bias, 
particularly as the modular nature of inference algorithms such as Gibb's sampling and variational approximation is suited to extending the model to handle different types of feedback that give indications of some underlying preferences. 

The GP methods require $\mathcal{O}(P_n)$ steps, where $P_n$ is the number of pairs for 
user $n$. 
The method proposed by \cite{khan2014scalable} reduces this scaling issue by using a random sample of pairs at each iteration of their EM algorithm.
We use SVI to address scalability in a variational Bayesian framework. 
The modular nature of VB allows us to take advantage of models for feedback of different types
where the input values for each type of feedback do not directly correspond (e.g. explicit user ratings and number of clicks may have different values).
By using SVI, we provide a formal way to deal with scalability that comes with guarantees\cite{hoffman2013stochastic}.
We also estimate the output scale of the GPs, the latent factors, and item bias as part of the 
variational approximation. %not clear what the true advantage of this is?

We compare our work on Sushi-A dataset or against the method of \cite{khan2014scalable} to see if 
our modifications are actually useful. 

Factor analysis differs from PPCA in allowing only diagonal noise covariance matrices, making 
the observed variables conditionally independent given the latent variables. It also provides
a probabilistic treatment for inferring the latent features. %are we still using FA?

We also investigate whether argumentation preferences can be reduced to a simpler
clustering structure, which may be easier to learn with very sparse user data.

% possible extension: state variable to describe what was previously seen? This could relate to time 
% since argument seen, and can be converted to an input feature for the GP model: exp(-t). I think
% that learning length scale and output scale for this feature would work.

%%%%% New additions in March 2017 -- edited a little when moved from other paper in August

In most scenarios where we wish to make predictions about arguments, 
there is a very large number of input variables potentially associated with each argument in the dataset,
but very sparse observations of these variables. 
To illustrate this, consider a simple bag-of-words representation of the argument text, and a set
of click-data recording which actions each user took when presented with a choice between different pieces of text. 
Given a large vocabulary, the words present in an argument will be a very small subset of possible words. Users will likely see a subset of texts and the recorded choices will be a much smaller subset of 
the possible combinations of texts. 
To make predictions about unobserved preferences when presented with a new text with sparse data,
we require an abstraction from the raw input data, and thus seek a way to embed the texts into a space 
where texts with similar properties are placed close together. In the case of arguments, one property
that may determine whether texts should be close together is that they have similar levels of 
convincingness to similar types of people, in similar contexts. Our proposal therefore produces
a form of argument embedding, driven by convincingness.
%Other work on argument embeddings was carried out by \cite{???}. 
A similar approach to learning latent features, VBMDS, is proposed by \cite{soh2016distance} for learning embeddings using approximate Bayesian techniques, but does not use the embeddings for 
preference learning to find separate person and item embeddings and does not apply this to NLP problems.
Their proposal does, however, show how to combine points with and without side information -- our
input features -- to make predictions about low-dimensional embeddings for unseen data. 
The kernelized probabilistic matrix factorization (KPMF) \cite{zhou2012kernelized} 
proposes a similar approach to VBMDS using GP priors over latent dimensions, but with a simpler
MAP inference scheme, and different likelihood and distance functions. 
% see section 4.1 in soh2016distance for more related work in this area, such as GPLVM.

An important aspect of convincingness is the context in which an argument is made, particularly
as part of a dialogue. 
In our approach, this context can be represented as input variables that affect the item and person embeddings, where the variables encapsulate the previously seen arguments.
While out-of-scope of the present investigation, future work may investigate the best way to
determine novelty of an argument given a small number of variables representing previously seen arguments.
Another related avenue of improvement is to consider the structure of arguments to select 
argument components -- it may be important to consider not just novelty, but whether claims have 
sufficient support and premises are clearly linked to the claims they support or attack. 
Embedding this structure may require complex graph structures of claims and premises to be represented
as short vectors, and may therefore be a topic of future study. 


% % I think this is future work?
% Secondly, kernel functions are not typically learned
% or adapted to the data, which means that points with different features that commonly co-occur are
% not assigned high covariance, whereas it would be desirable to learn that commonly co-occurring features
% indicate similar target values. 
% A solution to this problem is to represent input features such as words using vectors of continuous values, i.e. word embeddings. This approach was proposed for performing GP regression on 
% text data by \cite{yoshikawa2015non}, who showed how to learn the word embeddings and map document
% distributions over word embeddings to points in a reproducing kernel Hilbert space. 
% % this is what we need for using probabilistic embeddings? Do current probabilistic/Gaussian embeddings
% % just try to infer expected embedding and use it as input to another method? If so, 
% % we could see if there is an improvement in using kernel embeddings of distributions. The kernel
% % embedding is quite simple actually -- just the expectation of the kernel value with respect to the 
% % uncertain variable. The challenge would be to turn this into point value that can be used as 
% % input to a NN that uses no explicit kernel function.... or do they do something equivalent?
% This approach can be used to obtain document embeddings from word embeddings.

% E.g. product review texts. Training data contains +ve reviews with word "good". Unlabelled data
%contains reviews where "good" and "excellent" co-occur --> generative model learns to associate 
%"excellent" with +ve reviews. A GP regression model with "good" and "excellent" as binary input features
% would not be able to learn to associate "excellent" with +ve reviews through co-occurrence, it would 
% rely on "good" being present. 


The latent features allow us to interpolate between items and people in a low-dimensional embedding space. 
A key question in this latent feature approach is how to model the deviation of individual 
preferences from that predicted by latent features common to multiple people (item deviations
can be modelled through an item mean function).
This deviation occurs when there is still entropy in a user's preferences given the latent features
because the latent features only describe patterns that are common to multiple users.
A simple approach is to allow additional noise with uniform variance at each data point, 
so that all preference patterns are represented by the latent feature vectors of items and people.
However, any individual preference patterns particular to one user must then be represented by additional
latent features that are not activated for any other users. 
An alternative is to use a personal model of preference deviation for each person. 
Given the input features of the items and any state variables relating to the person, 
this model can capture correlations in the deviation for different items for the same person. 
Both the latent person features and the individual noise model can also include any input features of 
the person that change over time, e.g. representing their state and the arguments they have 
previously seen. 
This individual noise model allows us to differentiate preference patterns that are specific to 
one user, when the input features may not otherwise be sufficient to distinguish these users. 

Whether an argument is persuasive or not is subjective \cite{lukin2017argument},
hence analysing which arguments a particular person or group of people finds convincing can tell us
about their opinions and influences.

\section{Identifying Common Patterns of Convincingness}\label{sec:model}

Differences with/notes on Houlsby method:
\begin{itemize}
  \item No common mean
  \item Houlsby uses: w as weights for each user on the D latent functions; w has a GP prior with the user features
  as the input space; 
  in our code, I think we call these weights y just to be confusing;  they use a standard normal noise distribution
  for all users and all items
  \item It seems unnecessary to extend their model with a personal GP over the noise because this could be captured 
  by the latent GPs; therefore it may be better to start with nfactors \leq npeople, then see if the Bayesian shrinkage
  effect reduces the number of active latent functions due to people having common traits; this means we reduce the 
  number of GPs we need
  \item It may also be unecessary to extend their model with a separate noise level per worker because noisiness could also be captured by the noise level of the latent dimensions themselves; however it is possible that allowing
  variations in noise levels could reduce the number of latent factors needed, since we no longer need a latent
  factor to represent noise?
  \item Common means for all workers could be unnecessary, since this too could be captured by a latent factor that
  is shared by most workers; however, if we want to determine the 'consensus' preference, we need to identify this 
  common factor and it may be useful to view the workers as deviating from this underlying mean; this makes the view 
  more similar to CBCC view, where latent factors correspond to community confusion matrices, and user weights 
  correspond to worker community weights; the latent factor GPs model bias in a given community but unlike the CBCC model, they take into account the features of the items; the weights also account for user features when available, unlike CBCC; all modelling of noise, bias levels is then done at community level as in CBCC and individuals are
  described by weights alone.
  \item novelty comes then in using this model for combining preferences from crowdsourcing to infer an underlying consensus or a ground truth (specified by training data in semi-supervised mode) 
  \item experiments therefore need to look at not just predicting individuals' preference labels, but also at
  predicting a ground truth from noisy labels; we continue to use MACE with lots of labels to define ground truth,
  and see what happens when we use only one label per pair
  \item also need to examine differences between results produced by MACE and by collab. pref learning
  \item combining labels does not fit personalisation narrative? Personalised models help correct individual biases
  when inferring ground truth; helps transfer from one person's preferences to a target set of preferences
  \item Thompson sampling for active selection of labels from the crowd? like the crowdsample dataset but we get
  to choose which worker?
  \item Changing the current code: preference components needs to use a single GP with a diagonal covariance 
  and pump the results correctly back into self.f
\end{itemize}


\subsection{Baseline methods}

\begin{itemize}
  \item Random: select a label at random
  \item Most common (MC): select the most common preference label from across the dataset
  \item No differentiation (ND): we do not model differences between workers. Labels are estimated by taking the average of other people's labels for the same preference pair. When there are no previous pairs available, select the most common preference label
  \item Gaussian process preference learning with no differentiation (GP-ND): learn a latent ranking function for the objects from pairwise preferences, ignoring differences between workers and features of the arguments. This provides a probabilistic variant of ND  
\end{itemize}

\subsection{Modelling Correlations Between Individuals}

Two main types of approach:
\begin{itemize}
  \item Factor analysis -- map the set of pairwise preferences to a low-dimensional embedding
  \item Clustering -- assumes that people fall into distinct preference clusters, or can be modelled as a mixture of several archetypes
\end{itemize}

Specific methods to test can be split into several types. First,
we can run different clustering methods on the training data, 
then predict a worker's label by taking the mean of the other cluster members. 
When the no members of the cluster have labelled the pair, we predict using the most common label.
This method is applied to several clustering algorithms:
\begin{itemize}
   \item Affinity propagation (AP-mean)
   \item Gaussian mixture model, using most probable cluster assignment (GMM-mean)
   \item Gaussian mixture model, using cluster assignments weighted by probability (GMM-WM)
\end{itemize}

A similar approach can be taken with dimensionality reduction techniques, where we can use K-nearest neightbours (in this case, few workers label each pair, so we choose k=1 and use MC when no workers have labelled the current instance?):
\begin{itemize}
   \item Factor analysis with K-nearest neighbours (FA-KNN)
\end{itemize}
Alternatively, we can take a weighted average of the other labels for a pair, where the weights are based on inverse distance from the worker in question in the embedded space:
\begin{itemize}
   \item Factor analysis with an inverse distance-weighted mean (FA-weighted)
\end{itemize}
The distance function can be optimised, which leads to proposing more sophisticated methods...

\section{Bayesian Preference Learning Model}

The model introduced in \cite{houlsby2012collaborative} combines preference learning with matrix factorisation 
to identify latent features of items and users that affect their preferences. This allows for a collaborative filtering effect, whereby users with similar preferences on a set of observed items are assumed to have similar 
preferences for other items with similar features. This allows us to make better predictions about the unobserved preferences of a given user when we have seen preferences of a similar user.

The method presented in \cite{houlsby2012collaborative} uses a combination of expectation propagation (EP) and variational Bayes (VB). Since the inference steps require inverting a covariance matrix, this method scales with 
$\mathcal{O}(N^3)$ and is therefore impractical for large datasets. For our modified version of this method, we improve scalability by using stochastic variational inference to infer the complete model. 
The variational approximation to the posterior is given by...

The variational inference algorithm maximises a lower bound on the log marginal likelihood:
\begin{eqnarray}
  \mathcal{L} = \sum_{i=1}^N \mathbb{E}[ \log p(t_i | x_{i,1}, x_{i,2}, \bs f) ] + \nonumber\\
  \sum_{u=1}^U  \mathbb{E}\left[ \log \frac{p(\bs f_u | \bs w \bs y_u, \bs K_{f,u} / s_{f,u} )}{q(\bs f_u)}\right] + \nonumber\\
  \sum_{c=1}^C \mathbb{E}\left[\log\frac{p(\bs w_c | \bs 0, \bs K_w / s_{w,c} ) }{q(\bs w_c) } \right] + \nonumber\\
  \sum_{c=1}^C \mathbb{E}\left[\log\frac{p(\bs y_c | \bs 0, \bs K_y / s_{y,c} ) }{q(\bs y_c) } \right] + \nonumber\\
  \mathbb{E}\left[\log\frac{p(\bs t | \bs \mu, \bs K_{t} / s_t ) }{q(\bs t)} \right] + \nonumber\\
  \sum_{u=1}^U \mathbb{E}\left[\log\frac{p(s_{f,u} | a_{f,u}, b_{f,u})}{q(s_{f,u})}\right] + \nonumber\\
  \sum_{d=1}^D \mathbb{E}\left[\log\frac{p(s_{w,d} | a_{w,d}, b_{w,d})}{q(s_{w,d})}\right] +\nonumber\\
  \sum_{d=1}^D \mathbb{E}\left[\log\frac{p(s_{y,d} | a_{y,d}, b_{y,d})}{q(s_{y,d})}\right] 
\end{eqnarray}
where $t_i$ is the preference label for the $i$th pair, 

To perform feature selection with large numbers of features, we introduce 
an automatic relevance determination
(ARD) approach that uses the gradient of the lower bound on the log marginal likelihood to optimise the kernel length-scales using the L-BFGS-B method\cite{??}. The gradient is given by:
\begin{eqnarray}
\nabla\mathcal{L} = \left[ \frac{\partial \mathcal{L}}{\partial l_{w,1}}, ...,  \frac{\partial \mathcal{L}}{\partial l_{w,D_w}},  \frac{\partial \mathcal{L}}{\partial l_{y,1}}, ...,  \frac{\partial \mathcal{L}}{\partial l_{y,D_y}} \right], &&\\
\frac{\partial \mathcal{L}}{\partial l_{w,d}} = \frac{\partial}{\partial l_{w,d}} 
\sum_{u=1}^U  \mathbb{E}\left[\log\frac{p(\bs f_u | \bs w \bs y_u, \bs K_{f,u} / s_{f,u} )}{q(\bs f_u)}\right] + \nonumber && \\
\sum_{c=1}^C \mathbb{E}\left[\log\frac{p(\bs w_c | \bs 0, \bs K_w / s_{w,c} ) }{q(\bs w_c) } \right] - \nonumber&&\\
\sum_{u=1}^U \mathbb{E}\left[\log q(s_{f,u})\right] - \sum_{d=1}^D \mathbb{E}\left[\log q(s_{w,d})\right] +\nonumber&&\\
= 0.5 (\hat{f}_u - wy_u)^T \bs K_{f,u}^{-1} \frac{\partial \bs K}{\partial \log l_{w,d}} \hat{s}_{f,u} \bs K_{f,u}^{-1} (\hat{f}_u - wy_u) \nonumber\\
- 0.5\mathrm{tr}\left( (\bs K_{f,u}^{-1} - \frac{\bs C^{-1}}{\hat{s}_{f,u}} ) \frac{\partial \bs K_{f,u}}{\partial \log l_{w,d}} \right)\nonumber&&\\
\frac{\partial \mathcal{L}}{\partial l_{y,d}} = &&\\
\end{eqnarray}
where $l_{w,d}$ is a length-scale used for all the GPs over item features. The implicit terms are zero when the VB algorithm has converged.

\section{Experiments}\label{sec:expts}

In the first set of experiments we evaluate the baselines and the different methods for modelling correlations between workers' preferences. In the second set of experiments, we assess the value of different language features. Finally, the third experiment evaluates approaches that integrate both argument features and models of preference correlations.

Prior work on convincingness:
\begin{itemize}
 \item \cite{habernal2016argument} shows how to predict convincingness of arguments by training a NN 
 from crowdsourced annotations. 
 \item \cite{lukin2017argument} shows that persuasion is correlated with personality traits.
\end{itemize}

We build on this to show...
\begin{itemize}
 \item How we can predict convincingness for a specific user given only previous preferences and 
 preferences of others (collaborative filtering)
 \item How a combination of text and personality features improves predictions of convincingness
 \item That we can extract human-interpretable latent features in people and items,
 which improve performance over just using the input features.
\end{itemize}
This is all useful because we can use the approach to determine which features are worth 
obtaining, make predictions when data is sparse, and obtain data from users efficiently.

The steps to show this are:
\begin{enumerate}
  \item Show a table comparing the baselines, alternative collaborative filtering methods, 
  results from \cite{habernal2016argument}, and unsupervised method
  \item Add in results when using the input information with out method
  \item Show a table comparing the baselines, alternative collaborative filtering methods, 
  results from \cite{lukin2017argument}, and unsupervised method
  \item Add in results using item information, person information and both
  \item Visualise latent features?
  \item Table showing importance of input features
  \item Add results with lower confidence items excluded to the tables in 1-4. We can also plot the
  effect of confidence threshold on our results and on the rival methods.
  \item Add in Bier/cross entropy -- may need to rerun the original code from the previous papers?
  \item Run \cite{habernal2016argument} and my complete method with reduced data -- check accuracy as it increases. Use confidence cut-off from previous results.
  \item Simple active learning approach selecting the most uncertain data point (this will be due to 
  uncertainty about a person, an item with too little data, or disagreement/stochasticity in the likelihood). The plot can be added to the previous results and should be run with rival methods.
\end{enumerate}

\subsection{Note on difficult tasks for crowdworkers}

Would it be possible to include some results from argumentext here that show how workers were unable to be consistent with each other when marking argument spans?

\section{Future Work}

The collaborative preference model can be adapted so that it can be trained using 
classification data, scores/ratings (a regression task), or a mixture of different 
observation types by applying a different likelihood. 
The core of the method is the abstraction of a latent function over items and people, 
dependent on latent features of items and people, with the ability to include side information 
and observed features.
Future work will therefore investigate the ability to learn from multiple types of labelled data, 
(rather than only using preference pairs).

A further direction for future work is to apply this model to transfer learning: 
instead of modelling different latent functions per person, we model latent functions per task. 
Tasks for which the target function follows a similar pattern would then share information in 
a collaborative manner, so that training data for one task can inform similar tasks. This may be 
useful when data is limited, e.g. when performing domain adaptation. In the latter case, there would
need to be sufficient similarity between the features of the texts that are being classified for 
the collaborative effect to take place. 
For example, in argument mining, we may have several training datasets from different topics,
which can be used to learn a model of argument convincingness. Applying a collaborative model would
identify topics with common latent features, which would inform predictions on the target domain 
in parts of the feature space with no training data.


%%%%%%%%%%%%%%%%%%%%%%%%%%%%%%%%%%%%%%%%%%%%%%%%%%%%%%%%%%%%%%%%%%%%%%%%%%%%%%%%

% use section* for acknowledgment
%\section*{Acknowledgments}

% \addcontentsline{toc}{chapter}{Bibliography}
%\bibliographystyle{apalike}
\bibliographystyle{IEEEtran}
\bibliography{simpson_personalized_argumentation}

\end{document}
