\documentclass[11pt]{article}
\usepackage{acl2016}
\usepackage{times}
\usepackage{url}
\usepackage{latexsym}

\usepackage{graphicx}

\usepackage[fleqn]{amsmath}
\usepackage{amssymb}
\usepackage{amstext}
\usepackage{amsfonts}
\usepackage{amsthm}

\usepackage{cite}
\usepackage{algorithm2e}
\usepackage{array}
\usepackage[caption=false,font=footnotesize]{subfig}
\usepackage{url}
\usepackage{tabularx}

\newcommand{\bs}{\boldsymbol}  
\newcommand{\wrtd}{\mathrm{d}}

\makeatletter
\makeatother %some sort of hack related to the symbol @

\DeclareMathOperator*{\argmax}{\arg\!\max\!} %argmax operator

%%%%%%%%%%%%%%%%%%%%%%%%%%%%%%%%%%%%%%%%%%%%%%%%%%%%%%%%%%%%%%%%%

\title{ 
Personalised Models of Argument Convincingness
}

\author{\IEEEauthorblockN{Anonymous}
\IEEEauthorblockA{Anonymous, \\
Anonymous \\
Email: anonymous}
\and
\IEEEauthorblockN{Anonymous}
\IEEEauthorblockA{Anonymous, \\
Anonymous \\
Email: anonymous}
\and
\IEEEauthorblockN{Anonymous}
\IEEEauthorblockA{Anonymous, \\
Anonymous \\
Email: anonymous}
}

\begin{document}

\maketitle

\begin{abstract}
\end{abstract}

% For peer review papers, you can put extra information on the cover
% page as needed:
% \ifCLASSOPTIONpeerreview
% \begin{center} \bfseries EDICS Category: 3-BBND \end{center}
% \fi
%
% For peerreview papers, this IEEEtran command inserts a page break and
% creates the second title. It will be ignored for other modes.
%\IEEEpeerreviewmaketitle

%%%%%%%%%%%%%%%%%%%%%%%%%%%%%%%%%%%%%%%%%%%%%%%%%%%%%%%%%%%%%%%%%

\section{Introduction}\label{sec:intro}

Arguments are intended to persuade the audience of a particular point of view and 
are an important way for humans to reason about controversial topics~\cite{mercier2011humans}. 
The amount of argumentative text on any chosen subject can, however, overwhelm a reader.
Consider the scale of historical text archives 
or social media platforms with millions of users.
Automated methods could help readers overcome this challenge 
 by identifying high-quality, persuasive arguments from both sides of a debate. 
 
Theoretical approaches for assessing argument quality have proved difficult to apply to everyday arguments~\cite{boudry2015fake}.
Empirical machine learning approaches instead train models using example judgments of arguments,
such as those shown in Figure \ref{fig:argument_examples}.
%how does one assign a numerical convincingness score to each argument? 
%arguments paired with human judgments of their convincingness.
%approaches have recently shown success in assessing quality, convincingness or 
%persuasiveness \cite{habernal2016argument,wei2016post,wei2016preliminary,persing2017can} 
%Typically, machine learning approaches require examples of arguments paired with judgments of their convincingness.
\begin{figure}
\textbf{Topic:} ``William Farquhar ought to be honoured as the rightful founder of Singapore". \\
\textbf{Stance:} ``No, it is Raffles!" \\
\textbf{Argument 1:}  
HE HAS A BOSS(RAFFLES) HE HAS TO FOLLOW HIM AND NOT GO ABOUT DOING ANYTHING ELSE... \\
\textbf{Argument 2:} 
Raffles conceived a town plan to remodel Singapore into a modern city. The plan consisted of separate areas for different...\\
\textbf{Crowdsourced labels:} \{$2 \succ 1$, $1 \succ 2$, $2 \succ 1$\} 
\caption{Example argument pair from an online debate.}
\label{fig:argument_examples}
\end{figure}
Previous approaches to obtaining such judgments include training annotators to assign scores from 1-6~\cite{persing2017can}, 
asking annotators for simple binary or three-class categories~\cite{wei2016preliminary}, and
aggregating binary votes from multiple people~\cite{wei2016post,tan2016winning}.
However, these approaches are limited by the cost of training annotators, 
a highly restricted set of categories, or %that restricts filtering or sorting of arguments, or
the need for multiple annotators per document.

An alternative way to judge arguments is to compare them against one another~\cite{habernal2016argument}.
When comparing the arguments in Figure \ref{fig:argument_examples}, we may judge that argument 1 is less convincing due to its writing style, whereas argument 2 
presents evidence in the form of historical events.
Pairwise comparisons such as this are known to place less cognitive burden on human annotators than  
choosing a numerical rating and allow fine-grained sorting of items that is not possible with categorical labels
~\cite{kendall1948rank,kingsley2006preference}.
Unlike numerical ratings, pairwise comparisons are not affected by different annotators' biases
toward high, low or middling values, or an individual's bias changing over time.

In practice, we face a data acquisition bottleneck when encountering new domains or audiences.
For example, neural network methods typically require datasets with 
many thousands of hand-labeled examples to perform well~\cite{srivastava2014dropout,collobert2011natural}.
One solution is to employ multiple non-specialist annotators at low cost (\emph{crowdsourcing}), 
but this requires quality control techniques to account for errors.
Another source of data are the actions of users of a software application, which can be interpreted as pairwise judgments~\cite{joachims2002optimizing}. For example, when a user clicks on an argument in a list it can be interpreted
 as a preference for the selected argument over more highly-ranked arguments.
However, the resulting pairwise labels are likely to be a very noisy indication of preference.

In this paper, we develop a Bayesian approach to learn from noisy pairwise preferences
based on Gaussian process preference learning (GPPL)~\cite{chu2005preference}.
We model argument convincingness as a function of textual features, including word embeddings,
and develop an inference method for GPPL that scales to realistic dataset sizes using stochastic variational inference (SVI) ~\cite{hoffman2013stochastic}. % and make our software available at...
Using datasets provided by Habernal and Gurevych~\shortcite{habernal2016argument},
we show that our method outperforms the previous state-of-the-art for ranking arguments by convincingness 
and identifying the most convincing argument in a pair. 
%We also show the benefit of combining linguistic features with word embeddings.
Further experiments show that our approach is particularly advantageous with small, noisy datasets, 
and in an active learning set-up.
Our software is publicly available\footnote{\url{https://github.com/ukplab/tacl2018-preference-convincing}}.

The rest of the paper is structured as follows.
Section \ref{sec:related} reviews related work on argumentation,
then Section \ref{sec:bayesian} motivates the use of Bayesian methods by discussing their successful applications in NLP.
In Section \ref{sec:pref_learning}, we review preference learning methods and then Section \ref{sec:model}
describes our scalable Gaussian process-based approach.
Section \ref{sec:expts} presents our evaluation, 
comparing our method to the state-of-the art and testing with noisy data and active learning.
Finally, we present conclusions and future work.

\section{Identifying Convincing Arguments}\label{sec:related}

% Recently, Habernal and Gurevych~\shortcite{habernal2016makes} analysed reasons provided by annotators for why one argument is more convincing than another. In this paper we assume that explicit reasons are not provided. 
Lukin et al.~\shortcite{lukin2017argument} demonstrated that an audience's personality and prior stance affect
an argument's persuasiveness, but they were unable to predict belief change to a high degree of accuracy.
Related work has shown how persuasiveness is also affected by the sequence of arguments in a discussion 
~\cite{tan2016winning,rosenfeld2016providing,monteserin2013reinforcement},
but this work focuses on predicting salience of an argument given the state of the debate,
 rather than the qualities of arguments.

Wachsmuth et al.~\shortcite{wachsmuth2017argumentation} recently showed
 that relative comparisons of argument convincingness correlate with theory-derived quality ratings.
Habernal and Gurevych~\shortcite{habernal2016argument} established datasets
containing crowdsourced pairwise judgments of convincingness for arguments taken from online discussions. 
Errors in the crowdsourced data were handled by determining gold labels using the MACE algorithm~\cite{hovy2013learning}.
The gold labels were then used to train SVM and bi-directional long short-term memory (BiLSTM) classifiers to predict pairwise
labels for new arguments. 
The gold labels were also used to construct a directed graph of convincingness, which was input to PageRank 
to produce scores for each argument. 
These scores were then used to train SVM and BiLSTM regression models.
A drawback of such pipeline approaches is that they are prone to error propagation~\cite{chen2016joint},
and consensus algorithms, such as MACE, require multiple crowdsourced labels for each argument pair, 
which increases annotation costs.

\section{Bayesian Methods for NLP}\label{sec:bayesian}

When faced with a lack of reliable annotated data, 
Bayesian approaches have a number of advantages.
Bayesian inference provides a mathematical framework for combining multiple observations
with prior information. 
Given a model, $M$, and observed data, $D$, we apply Bayes' rule
to obtain a \emph{posterior distribution} over $M$, which can be used to make predictions 
about unseen data or latent variables:
\begin{equation}
  P(M|D) = \frac{P(D|M)P(M)}{P(D)},
  \label{eq:bayesrule}
\end{equation}
where $P(D|M)$ is the likelihood of the data given $M$, and $P(M)$ is the model prior.
If the dataset is small, the posterior remains close to the prior, so the model 
does not assume extreme values given a small training sample.
%which reduces overfitting.
Rather than learning a posterior, neural network training typically selects model parameters that maximize the likelihood, 
so they are more prone to overfitting with small datasets, which can reduce performance~\cite{xiong2011bayesian}.

Bayesian methods can be trained using unsupervised or semi-supervised learning
to take advantage of structure in unlabeled data when labeled data is in short supply.
Popular examples in NLP are
Latent Dirichlet Allocation (LDA)~\cite{blei2003latent}, which is used for topic modelling,
and its extension, the hierarchical Dirichlet process (HDP)~\cite{teh2005sharing}, which learns the number of topics rather than requiring it to be fixed a priori.
Semi-supervised Bayesian learning
has also been used to achieve state-of-the-art results for semantic role labelling~\cite{titov2012bayesian}.

We can combine independent pieces of weak evidence using Bayesian methods through the likelihood.
For instance, a Bayesian network can be used to infer attack relations between arguments by combining votes for acceptable arguments from different people~\cite{kido2017}.
Other Bayesian approaches combine crowdsourced annotations to train a sentiment classifier
without a separate quality control step~\cite{simpson2015language,felt2016semantic}.

Several successful Bayesian approaches in NLP make use of Gaussian processes (GPs), which are 
distributions over functions of input features. 
GPs are nonparametric, meaning they can model highly nonlinear functions by
allowing function complexity to grow with the amount of data~\cite{rasmussen_gaussian_2006}.
They account for model uncertainty when extrapolating from sparse training data
and can be incorporated into larger graphical models.
Example applications include analyzing the relationship between a user's impact on Twitter 
and the textual features of their tweets~\cite{lampos2014predicting}, 
predicting the level of emotion in text~\cite{beck2014joint},
and estimating the quality of machine translations given source and translated texts~\cite{cohn2013modelling}.

\section{Preference Learning}\label{sec:pref_learning}

% GP-based approach versus Bradley/Terry/Luce/MPM method or Mallows models?
Our aim is to develop a Bayesian method for identifying convincing arguments 
given their features, which can be trained on noisy pairwise labels.
Each label, $i \succ j$, states that an argument, $i$, is more convincing than argument, $j$. 
This learning task is a form of \emph{preference learning}, which can be addressed in several ways.
% rank or score arguments in terms of convincingness,
% or predict which item in a subset the user finds more convincing. 
%Given a ranking over items, it is possible to determine the pairwise preferences, but 
A simple approach is to use a generic classifier by
obtaining a single feature vector for each pair in the training and test datasets,
either by concatenating the feature vectors of the items in the pair, 
or by computing the difference of the two feature vectors, as in SVM-Rank~\cite{joachims2002optimizing}. 
However, this approach does not produce ranked lists of convincing arguments without predicting a large number of pairwise labels, nor give scores of convincingness. 

Alternatively, we could learn an ordering over arguments directly using Mallows models~\cite{mallows1957non},
which define distributions over permutations.  
Mallows models can be trained from pairwise preferences 
~\cite{lu2011learning}, but inference is usually costly
since the number of possible permutations is $\mathcal{O}(N!)$, 
where $N$ is the number of arguments. 
Modeling only the ordering does not allow us to quantify 
the difference between arguments at similar ranks.

To avoid the problems of classifier-based and permutation-based methods, 
we propose to learn a real-valued convincingness function, $f$, that takes argument features as input
and can be used to predict rankings, pairwise labels, or ratings for individual arguments.
There are two well established approaches for mapping pairwise labels to real-valued scores: 
the Bradley-Terry-Plackett-Luce model~\cite{bradley1952rank,luce1959possible,plackett1975analysis}
and the Thurstone-Mosteller model~\cite{thurstone1927law,mosteller2006remarks}.
Based on the latter approach, 
Chu and Ghahramani~\shortcite{chu2005preference} introduced 
Gaussian process preference learning (GPPL), 
a Bayesian model that can tolerate errors in pairwise training labels
and gains the advantages of a GP for learning nonlinear functions from sparse datasets.
However, the inference method proposed by Chu and Ghahramani~\shortcite{chu2005preference} 
has memory and computational costs that scale with $\mathcal{O}(N^3)$,
making it unsuitable for real-world text datasets. 
The next section explains how we use recent developments in inference methods 
to develop scalable Bayesian preference learning for argument convincingness.
\section{Bayesian Preference Learning for Crowds}\label{sec:model}

%%%% Notes

% Title or name of the model: 
% -- cannot decide this until we get most of the paper complete: will emphasis be on crowds? distilling ground truth
% from noisy sources (Bayesian preference learning for fusing unreliable sources)? user preferences/collaborative filtering?
% -- need a new name to differentiate from Houlsby et al. and Khan et al.?
% -- what are the differences in the model? Let's get the model written up.
% -- should also be some buzzword or word to generate interest: 
%    -- 'variational' is on the up, could be used in paper title
%    -- 'stochastic variational' is also on the up
%    -- 'crowdsourcing' on the way down as at 2010 level
%    -- 'gaussian process' on the way up
%    -- 'matrix factorization' kind of on the way up
%    -- 'scalable' on the way up
%    -- 'interactive learning' on the way down
%    -- 'preference learning' flattish, may be on way up slowly
% -- aimed at crowdsourcing problems (uses a common mean function as consensus?)
% -- other parameters for importance of features?
% -- combined preference learning? Preference aggregation? Collaborative crowdsourced preference learning? Bayesian preference learning for crowds? another word for 'multi-user' or 'many users and many items' vs. crowds?

% TO ADD: Why does the variance in f cancel out when predicting the probability of a pairwise label?
% TO ADD: Why does \sigma disappear if we learn the output scale.

% We also estimate the output scale of the GPs, the latent factors, and item bias as part of the 
% variational approximation allowing us to estimate these parameters in a Bayesian manner without 
% resorting to maximum likelihood approaches.

% mention how the noise model deals with inconsistencies in preferences

% \begin{enumerate}
% \item What are the benefits of Bayesian methods and Gaussian processes in particular?
% \item The proposal by \cite{chu2005preference} shows how the advantages of a Bayesian
% approach can be exploited for preference learning by modifying the observation model

% Extensions:
% -- how do we replace the GP with a NN?
% -- would this move us from a Bayesian to an ML solution?
% -- maybe save this for future work? Or add a few lines if we can make it fit with the theme of the paper.
% -- another is to replace the fixed number of clusters with a CRP, then the whole thing can be nonparameteric preference learning with crowds.

\subsection{Modeling Pairwise Preferences}

% Include the case for one user -- preferences may depend on a number of observed features.

% this should be introduced in section 1.
A pairwise comparison between items $i$ and $j$ has a binary label that is either $i \succ j$
if $i$ is preferred to $j$, or $i \prec j$ is $j$ is preferred to $i$.

\subsection{Latent Factors: Bayesian Matrix Factorization}

% 0. Consider multiple users, each with a set of observed features.
% 1. imagine augmenting the observed features with a number of latent features (this is kind of what Khan et al. do)
% 2. imagine that the latent features relate preference function values between items and users using the smallest number of features
% 3. hence the observed features are not needed given the latent features, but we can create a hierarchical model where the latent features depend on the observed ones if available.

\subsection{The complete model}

% joint distribution
% notes about problems with inference.

\subsection{OLD}

%%%% The preference likelihood model

Following Chu and Ghahramani~\citeyear{chu2005preference}, 
we model the relationship between a latent preference function, $f$,
and each observed pairwise label, $v_k \succ u_k$, where $k$ is an index into a list of 
$P$ pairs, as follows:
\begin{flalign}
& p( v_k \succ u_k | f(v_k), f(u_k), \delta_{v_k}, \delta_{u_k} ) & \nonumber\\
& \hspace{1.5cm} = \begin{cases}
 1 & \text{if }f(v_k) + \delta_{v_k} \geq f(u_k) + \delta_{u_k} \\
 0 & \text{otherwise,}
 \end{cases} &
\end{flalign}
where $\delta_i \sim \mathcal{N}(0, 1)$ is Gaussian-distributed noise. 
The noise term allows for variations in the observed preferences, which may occur if 
different annotators disagree or change their minds, or if
the preferences are derived from noisy implicit data such as clicks streams.
We deviate from Chu and Ghahramani~\citeyear{chu2005preference} by assuming $\delta_i$
has variance $\sigma=1$, and instead scale the function $f$ relative to this. This formulation
is equivalent but is more convenient for variational inference. 
We marginalise the noise terms to obtain the preference likelihood:
\begin{flalign}
& p( v_k \succ u_k | f(v_k), f(u_k) ) = \Phi\left( \frac{f(v_k) - f(u_k)}{\sqrt 2} \right), & \label{eq:pl}
\end{flalign}
where $\Phi$ is the cumulative distribution function of a standard Gaussian distribution.
For the latent function , $f$, we assume a Gaussian process prior: $f \sim \mathcal{GP}(0, k_{\theta}/s)$, where 
$k_{\theta}$ is a kernel function with hyper-parameters $\theta$, 
and $s \sim \mathcal{G}(a_0, b_0)$ is an inverse scale parameter %controlling the noise level the level of noise in the function and has 
drawn from a gamma prior with shape $a_0$ and scale $b_0$.
The kernel function controls the correlation between values of $f$ at different points in the feature space.

\section{Experiments}\label{sec:expts}

In the first set of experiments we evaluate the baselines and the different methods for modelling correlations between workers' preferences. In the second set of experiments, we assess the value of different language features. Finally, the third experiment evaluates approaches that integrate both argument features and models of preference correlations.

Prior work on convincingness:
\begin{itemize}
 \item \cite{habernal2016argument} shows how to predict convincingness of arguments by training a NN 
 from crowdsourced annotations. 
 \item \cite{lukin2017argument} shows that persuasion is correlated with personality traits.
\end{itemize}

We build on this to show...
\begin{itemize}
 \item How we can predict convincingness for a specific user given only previous preferences and 
 preferences of others (collaborative filtering)
 \item How a combination of text and personality features improves predictions of convincingness
 \item That we can extract human-interpretable latent features in people and items,
 which improve performance over just using the input features.
\end{itemize}
This is all useful because we can use the approach to determine which features are worth 
obtaining, make predictions when data is sparse, and obtain data from users efficiently.

The steps to show this are:
\begin{enumerate}
  \item Show a table comparing the baselines, alternative collaborative filtering methods, 
  results from \cite{habernal2016argument}, and unsupervised method
  \item Add in results when using the input information with out method
  \item Show a table comparing the baselines, alternative collaborative filtering methods, 
  results from \cite{lukin2017argument}, and unsupervised method
  \item Add in results using item information, person information and both
  \item Visualise latent features?
  \item Table showing importance of input features
  \item Add results with lower confidence items excluded to the tables in 1-4. We can also plot the
  effect of confidence threshold on our results and on the rival methods.
  \item Add in Bier/cross entropy -- may need to rerun the original code from the previous papers?
  \item Run \cite{habernal2016argument} and my complete method with reduced data -- check accuracy as it increases. Use confidence cut-off from previous results.
  \item Simple active learning approach selecting the most uncertain data point (this will be due to 
  uncertainty about a person, an item with too little data, or disagreement/stochasticity in the likelihood). The plot can be added to the previous results and should be run with rival methods.
\end{enumerate}

\subsection{Note on difficult tasks for crowdworkers}

Would it be possible to include some results from argumentext here that show how workers were unable to be consistent with each other when marking argument spans?

\section{Conclusions and Future Work}

We presented a novel Bayesian approach to predicting argument convincingness from pairwise labels using
Gaussian process preference learning (GPPL).
Using recent advances in approximate inference, we developed a scalable algorithm for GPPL 
that is suitable for large NLP datasets.
Our experiments demonstrated that our method significantly outperforms the state-of-the-art
on a benchmark dataset for argument convincingness, 
particularly when noisy and conflicting pairwise labels are used in training.
Active learning experiments showed that GPPL is an effective model for cold-start situations 
and that the convincingness of Internet arguments can be predicted reasonably 
well given only a small number of samples.
The results also showed that linguistic features and word embeddings provide complementary information,
and that GPPL can be used to automatically identify relevant features.

Future work will evaluate our approach on other NLP tasks 
where reliable classifications may be difficult to obtain, 
such as learning to classify text from implicit user feedback~\cite{joachims2002optimizing}.
We also plan to investigate training the GP using absolute scores in combination with pairwise labels.


%%%%%%%%%%%%%%%%%%%%%%%%%%%%%%%%%%%%%%%%%%%%%%%%%%%%%%%%%%%%%%%%%%%%%%%%%%%%%%%%

% use section* for acknowledgment
%\section*{Acknowledgments}

% \addcontentsline{toc}{chapter}{Bibliography}
%\bibliographystyle{apalike}
\bibliographystyle{IEEEtran}
\bibliography{simpson_personalized_argumentation}

\end{document}
