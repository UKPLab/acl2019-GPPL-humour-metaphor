\section{Future Work}

The collaborative preference model can be adapted so that it can be trained using 
classification data, scores/ratings (a regression task), or a mixture of different 
observation types by applying a different likelihood. 
The core of the method is the abstraction of a latent function over items and people, 
dependent on latent features of items and people, with the ability to include side information 
and observed features.
Future work will therefore investigate the ability to learn from multiple types of labelled data, 
(rather than only using preference pairs).

A further direction for future work is to apply this model to transfer learning: 
instead of modelling different latent functions per person, we model latent functions per task. 
Tasks for which the target function follows a similar pattern would then share information in 
a collaborative manner, so that training data for one task can inform similar tasks. This may be 
useful when data is limited, e.g. when performing domain adaptation. In the latter case, there would
need to be sufficient similarity between the features of the texts that are being classified for 
the collaborative effect to take place. 
For example, in argument mining, we may have several training datasets from different topics,
which can be used to learn a model of argument convincingness. Applying a collaborative model would
identify topics with common latent features, which would inform predictions on the target domain 
in parts of the feature space with no training data.
